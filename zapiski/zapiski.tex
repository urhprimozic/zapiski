\documentclass[a4paper,12pt]{article}

\usepackage[slovene]{babel}
\usepackage{amsfonts,amssymb,amsmath}
\usepackage[utf8]{inputenc}
\usepackage[T1]{fontenc}
\usepackage{lmodern}
\usepackage{graphicx}
\usepackage{tikz}



\def\N{\mathbb{N}} % mnozica naravnih stevil
\def\Z{\mathbb{Z}} % mnozica celih stevil
\def\Q{\mathbb{Q}} % mnozica racionalnih stevil
\def\R{\mathbb{R}} % mnozica realnih stevil
\def\C{\mathbb{C}} % mnozica kompleksnih stevil
\def\dif{\mathrm{d}}
\def\G{\mathcal{G}}

\newcommand{\p}{\to} % povezava med vozljišči

\newcommand{\norm}[1]{\lVert #1 \rVert _\infty}
\newcommand{\n}[1]{\lVert #1 \rVert }


\newcommand{\scalar}[2]{\langle #1 \, , #2 \rangle}
\def\qed{$\hfill\Box$}   % konec dokaza
\def\qedm{\qquad\Box}   % konec dokaza v matematičnem načinu
\newtheorem{izrek}{Izrek}
\newtheorem{trditev}{Trditev}
\newtheorem{posledica}{Posledica}
\newtheorem{lema}{Lema}
\newtheorem{pripomba}{Pripomba}
\newtheorem{definicija}{Definicija}
\newtheorem{zgled}{Zgled}


\author{Urh Primožič}
\title{Funkcijski grafi \\
\small{zapiski}}
\begin{document}
\maketitle
\begin{abstract}
    V tem dokumentu so opisani zanimivi izsledki poletnih raziskovanj.
\end{abstract}

\section{Uvod}
TODO motivacija 

\section{Definicije}

\begin{definicija}
    Naj bo $f \colon X \to X$ poljubna funkcija. 
    Definiramo usmerjen graf $\G_f=(X,E)$, $u \p v$, če $v = f(u)$. Rečemo, da je $\G_f$ funkcijski graf funkcije $f$.
\end{definicija}

\begin{definicija}
    Množica funkcijskih grafov nad $X$ je 
    $\G = \{\G_f \mid f \in X^X \}$.
\end{definicija}

\begin{definicija}
    Grafa $G=(V,E)$ in $G' = (V', E')$ sta 
    si izomorfna, če obstaja bijekcija $\varphi \colon V \to V'$, 
    da za vsaka $ v,w \in V$ velja $v \p w \iff \varphi(v) 
    \p \varphi(w)$. Pišemo $G \approx G'$.
\end{definicija}
Izomorfizem je očitno ekvivalenčna relacija. Zato lahko definiramo 
množico ekvivalenčnih razredov $\G/_\approx$.
\begin{definicija}
    Definiramo $G$ nad $X$ kot kvocientno 
    množico $\G /_\approx$, kjer $\approx$ predstavlja 
    izomorfnost grafov.
\end{definicija}

Zanima me, če ima $G$ nad lepimi množicami lepe lastnosti. 

\section{Funkcijski grafi nad $\N$}
V tem razdelku študiram grafe, porojene s funkcijami nad podmnožicami naravnih števil.
\section{Vprašanja}
TODO opisan


\subsection{Ali je vsak funkcijski graf ravninski}
Glejmo Grafe nad $\N$. Ali so ravninski? Ali obstaja lepa karakterizacija za ravninske funkcijske grafe?
\subsection{Ali se na $G$ prenesejo algebraične lastnosti}
Vem že, da na $G$ nad $\N$ lahko štejemo (TODO). Kaj več?
\end{document}